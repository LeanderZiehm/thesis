\documentclass[12pt,a4paper]{article}

% ——— PACKAGES ———
\usepackage[utf8]{inputenc}
\usepackage[T1]{fontenc}
\usepackage[english]{babel}
\usepackage{graphicx}
\usepackage{geometry}
\usepackage{hyperref}
\usepackage{titlesec}
% \usepackage{mathptmx} % Times New Roman clone

% ——— PAGE LAYOUT ———
\linespread{1.5} % 1.5 line spacing
\geometry{left=2.54cm, right=2.54cm, top=2.54cm, bottom=2.54cm} % 1 inch margins

% ——— METADATA ———
\newcommand{\university}{Deggendorf Institute of Technology}
\newcommand{\faculty}{Faculty of Computer Science, Bachelor Artificial Intelligence}
% \newcommand{\papertitle}{Can Customer Behavior During the First 14 Days After Signup Predict Churn Within 90 Days, Using Logistic Regression?}
\newcommand{\papertitle}{To what extent does integrating CLV predictions into marketing automation improve customer engagement and reduce churn?}
\newcommand{\authorname}{Leander-Arun Ziehm}
\newcommand{\studentid}{22201349}
\newcommand{\supervisor}{Johannes Reisinger, M.Sc.}
\newcommand{\submissiondate}{September 1, 2025}

% ——— TITLE PAGE ———
\begin{document}
\begin{titlepage}
  \centering
  \vfill

  \includegraphics[width=9cm]{THD_Logo.jpg}\\[1cm]
  {\Large \university}\\
  {\large \faculty}\\[1.5cm]

  {\LARGE \textbf{Mini Exposé}}\\[1cm]
  {\LARGE \textbf{\papertitle}}\\[1.5cm]
  
  \begin{flushleft}
    \textbf{Student Name:} \authorname\\
    \textbf{Matriculation Number:} \studentid\\
    \textbf{Supervisor:} \supervisor\\
    \textbf{Start Date:} \submissiondate
  \end{flushleft}

  \vfill
\end{titlepage}

\begin{abstract}
Customer churn is a critical issue for subscription-based and transactional businesses, directly impacting revenue and growth. This thesis proposes to develop a rigorous, explainable, and actionable system for predicting Customer Lifetime Value (CLV) and preventing churn using company transactional and usage data. By leveraging classical statistical models such as BG/NBD and Gamma-Gamma, combined with rule-based churn heuristics and automated targeted incentives, the system aims to enable scalable, data-driven retention strategies. The project focuses on utilizing the existing company data store and product usage logs to build a transparent model that prioritizes high-value customers and optimizes incentive campaigns, while ensuring compliance with relevant data privacy regulations.
\end{abstract}

\section{Introduction}
Customer retention is a major driver of sustainable business growth, especially in highly competitive markets. While many companies invest in churn prediction models, the challenge lies in translating predictions into effective, scalable retention actions. This thesis will address this gap by designing a system that combines explainable CLV modeling with automated incentive delivery, prioritizing efforts on valuable customers at risk of churn.

The company maintains extensive transactional and usage data in its data warehouse, including purchase histories, subscription status, product interactions, and customer support records. This data provides the foundation for building robust predictive models and executing targeted retention campaigns.

\section{Research Objectives}
The main objectives of this thesis are:
\begin{itemize}
    \item To develop a statistically rigorous and explainable CLV prediction model using classical probabilistic methods (BG/NBD and Gamma-Gamma) on company transactional data.
    \item To define and implement rule-based heuristics for churn risk estimation using product usage and engagement data.
    \item To integrate CLV and churn risk scores into a prioritization framework for targeting customers with automated incentive email campaigns.
    \item To design, deploy, and evaluate the impact of incentive campaigns on churn reduction and revenue uplift.
    \item To ensure compliance with data privacy and marketing regulations applicable in the company’s jurisdiction.
\end{itemize}

\section{Literature Review}
Classical probabilistic models such as the BG/NBD (Beta Geometric / Negative Binomial Distribution) and Gamma-Gamma model have been widely adopted for CLV estimation \cite{fader2005counting,fader2007modeling}. These models offer interpretability and reliability, especially in non-contractual settings.

Churn prediction literature spans statistical methods, machine learning, and survival analysis \cite{verbeke2012predictive, verbeke2011new}. However, the practical integration of CLV and churn scores into actionable retention tactics remains an ongoing research challenge \cite{neslin2006defection}.

Automated marketing campaign design, including A/B testing and uplift modeling, provides a pathway to quantify the effectiveness of retention incentives \cite{radcliffe2007using}. Regulatory frameworks like GDPR impose important constraints on customer contact, necessitating privacy-aware system design \cite{voigt2017eu}.

\section{Methodology}
\subsection{Data Collection and Preparation}
The study will utilize:
\begin{itemize}
    \item \textbf{Transactional data}: purchase dates, amounts, customer identifiers, subscription start/end dates.
    \item \textbf{Usage data}: login frequency, feature usage metrics, session durations.
    \item \textbf{Support data}: customer service interactions and ticket logs.
\end{itemize}
Data cleaning and preprocessing will be conducted to construct customer-level summary statistics and time series.

\subsection{Modeling}
\begin{itemize}
    \item \textbf{CLV Modeling:} Using the \texttt{lifetimes} Python package, implement BG/NBD and Gamma-Gamma models to estimate future purchase frequency and monetary value.
    \item \textbf{Churn Heuristics:} Develop rule-based flags (e.g., inactivity periods, usage decline thresholds) to estimate churn risk in the absence of labeled churn data.
    \item \textbf{Integration:} Combine CLV and churn risk scores into a prioritization schema for targeted retention.
\end{itemize}

\subsection{Campaign Design and Automation}
Using the prioritization framework, design automated incentive email campaigns delivering personalized offers. Campaigns will be deployed via a marketing automation platform (e.g., Customer.io) and instrumented for A/B testing.

\subsection{Evaluation}
Evaluate campaign effectiveness via:
\begin{itemize}
    \item Churn rate changes between test and control groups.
    \item Revenue uplift and changes in CLV post-intervention.
    \item Statistical significance testing of retention effects.
\end{itemize}

\subsection{Compliance}
Review and document data privacy measures, including consent management and opt-out handling, ensuring GDPR and other applicable regulation compliance.

\section{Expected Contributions}
This research aims to:
\begin{itemize}
    \item Provide a transparent, statistically sound framework for CLV prediction in a practical business setting.
    \item Bridge the gap between churn prediction and actionable retention interventions.
    \item Demonstrate how classical models can be integrated with automated marketing to improve customer retention.
    \item Deliver a replicable blueprint for companies starting from minimal infrastructure.
\end{itemize}

\section{Timeline}
\begin{tabular}{p{0.25\linewidth} p{0.7\linewidth}}
Month 1 & Literature review, data acquisition, initial exploration \\
Month 2 & Development of CLV and churn heuristic models \\
Month 3 & Integration and segmentation framework design, setup of automated email campaigns and A/B testing framework, campaign deployment and data collection \\
Month 4 & Evaluation and analysis of results \\
Month 5 & Writing thesis draft \\
Month 6 & Revision and submission \\
\end{tabular}













\newpage

% ===== REFERENCES =====
\begin{thebibliography}{9}


\end{thebibliography}

\end{document}

