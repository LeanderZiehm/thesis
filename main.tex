\documentclass[12pt,a4paper]{article}

% ——— PACKAGES ———
\usepackage[utf8]{inputenc}
\usepackage[T1]{fontenc}
\usepackage[english]{babel}
\usepackage{graphicx}
\usepackage{geometry}
\usepackage{hyperref}
\usepackage{titlesec}
\usepackage{array}       % for p{} in tables
\usepackage{booktabs}    % better tables (optional)
\usepackage{enumitem}    % control list spacing (optional)

% ——— PAGE LAYOUT ———
\linespread{1.5} % 1.5 line spacing
\geometry{left=2.54cm, right=2.54cm, top=2.54cm, bottom=2.54cm} % 1 inch margins

% ——— METADATA ———
\newcommand{\university}{Deggendorf Institute of Technology}
\newcommand{\faculty}{Faculty of Computer Science, Bachelor Artificial Intelligence}
\newcommand{\papertitle}{Predicting Customer Behavior and Lifetime Value Using Explainable AI for Marketing Insights in a SaaS Platform}

\newcommand{\authorname}{Leander-Arun Ziehm}
\newcommand{\studentid}{22201349}
\newcommand{\supervisor}{Johannes Reisinger, M.Sc.}
\newcommand{\submissiondate}{September 1, 2025}

% ——— TITLE PAGE ———
\begin{document}

\begin{titlepage}
  \centering
  \vfill

  \includegraphics[width=9cm]{THD_Logo.jpg}\\[1cm]

  {\Large \university}\\
  {\large \faculty}\\[1.5cm]

  {\LARGE \textbf{Mini Exposé}}\\[1cm]
  {\LARGE \textbf{\papertitle}}\\[1.5cm]

  \begin{flushleft}
    \textbf{Student Name:} \authorname\\
    \textbf{Matriculation Number:} \studentid\\
    \textbf{Supervisor:} \supervisor\\
    \textbf{Start Date:} \submissiondate
  \end{flushleft}

  \vfill
\end{titlepage}

\clearpage
\pagenumbering{arabic}

\begin{abstract}
Customer behavior prediction is a strategic advantage for SaaS companies seeking to optimize growth, retention, and revenue. This thesis proposes the development of an explainable AI system to predict Customer Lifetime Value (CLV), likelihood of converting from trial to paid, and risk of churn from paid subscriptions. By integrating classical statistical models with machine learning and explainability techniques (e.g., SHAP values), the system will provide transparent and actionable insights for marketing and product teams. A key outcome of the project is an interactive dashboard that visualizes individual and segment-level predictions, enabling data-driven decision-making without relying on opaque black-box models or automated outreach. The system will utilize real customer data, while adhering to data protection regulations, and aims to create a replicable framework for AI-driven marketing analytics in B2B SaaS contexts.
\end{abstract}

\section{Introduction}
Customer retention is a major driver of sustainable business growth, especially in highly competitive markets. While many companies invest in churn prediction models, the challenge lies in translating predictions into effective, scalable retention actions. This thesis will address this gap by designing a system that combines explainable CLV modeling with automated incentive delivery, prioritizing efforts on valuable customers at risk of churn.

The company maintains extensive transactional and usage data in its data warehouse, including purchase histories, subscription status, product interactions, and customer support records. This data provides the foundation for building robust predictive models and executing targeted retention campaigns.

\section{Research Objectives}
The main objectives of this thesis are:
\begin{itemize}[leftmargin=*]
    \item To develop an explainable prediction system for key customer behaviors: conversion from trial to paid, churn risk, and expected CLV.
    \item To evaluate and compare interpretable models (e.g., logistic regression, decision trees) with more complex ML models (e.g., gradient boosting) using explainability tools.
    \item To construct a dashboard that visualizes individual-level and cohort-level predictions, along with feature importance and confidence indicators.
    \item To provide marketing and business stakeholders with actionable, transparent insights into customer behavior drivers.
    \item To ensure compliance with GDPR and data privacy best practices in the design and deployment of predictive tools.
\end{itemize}

\section{Literature Review}

\section{Methodology}

\subsection{Data Collection and Preparation}
The study will utilize:
\begin{itemize}[leftmargin=*]
    \item \textbf{Transactional data}: purchase dates, amounts, customer identifiers, subscription start/end dates.
    \item \textbf{Usage data}: login frequency, feature usage metrics, session durations.
    \item \textbf{Support data}: customer service interactions and ticket logs.
\end{itemize}
Data cleaning and preprocessing will be conducted to construct customer-level summary statistics and time series.

\subsection{Modeling}
\begin{itemize}[leftmargin=*]
    \item \textbf{CLV Modeling:} Using the \texttt{lifetimes} Python package, implement BG/NBD and Gamma-Gamma models to estimate future purchase frequency and monetary value.
    \item \textbf{Churn Heuristics:} Develop rule-based flags (e.g., inactivity periods, usage decline thresholds) to estimate churn risk in the absence of labeled churn data.
    \item \textbf{Integration:} Combine CLV and churn risk scores into a prioritization schema for targeted retention.
\end{itemize}

\subsection{Campaign Design and Automation}
Using the prioritization framework, design automated incentive email campaigns delivering personalized offers. Campaigns will be deployed via a marketing automation platform (e.g., Customer.io) and instrumented for A/B testing.

\subsection{Evaluation}
Evaluate campaign effectiveness via:
\begin{itemize}[leftmargin=*]
    \item Churn rate changes between test and control groups.
    \item Revenue uplift and changes in CLV post-intervention.
    \item Statistical significance testing of retention effects.
\end{itemize}

\subsection{Dashboard for Explainable AI}
A central deliverable will be a dashboard that integrates predictions from the AI models and presents them in an intuitive, marketing-friendly interface. The dashboard will include:
\begin{itemize}[leftmargin=*]
    \item Customer-level predictions: CLV, churn risk, and trial-to-paid likelihood.
    \item Global and local explainability: using SHAP values to highlight feature importance for each prediction.
    \item Cohort analytics: visual summaries of behavioral segments, e.g., high-value likely churners.
    \item Action suggestions: data-driven hypotheses to inform manual marketing actions.
\end{itemize}
Tools such as Flask with Python backend will be considered for implementation.

\section{Expected Contributions}
This research aims to:
\begin{itemize}[leftmargin=*]
    \item Deliver a transparent, explainable framework for predicting Customer Lifetime Value (CLV), churn risk, and trial-to-paid conversion in a SaaS environment.
    \item Demonstrate the integration of classical models and modern machine learning with explainability tools (e.g., SHAP) to support data-driven decision-making.
    \item Build an interactive dashboard that visualizes model predictions and their key drivers, enabling marketing and business teams to interpret and act on insights.
    \item Provide a replicable methodology for companies with limited infrastructure to adopt explainable AI for customer analytics.
    \item Ensure responsible use of customer data by incorporating privacy-preserving techniques and regulatory compliance (e.g., GDPR).
\end{itemize}

\section{Timeline}
\begin{tabular}{p{2cm}p{12cm}}
\toprule
Month 1 & Literature review, data acquisition, initial data cleaning and exploration \\
Month 2 & Modeling customer conversion, churn risk, and CLV using statistical and ML techniques \\
Month 3 & Evaluation of models, integration of explainability (e.g., SHAP, LIME), model selection \\
Month 4 & Build interactive dashboard with explainable AI outputs and marketing-relevant visualizations \\
Month 5 & Writing thesis draft with technical and business insights \\
Month 6 & Revision, feedback, and final submission \\
\bottomrule
\end{tabular}

\newpage

% ===== REFERENCES =====
\begin{thebibliography}{9}

\bibitem{quintero2024}
Into Quintero,  
\textit{Customer Lifetime Value as a Part of the Customer Journey},  
Bachelor’s Thesis, Bachelor of Business Administration, International Business,  
Tampere University of Applied Sciences, April 2024.  
Available: \url{https://www.theseus.fi/bitstream/handle/10024/852910/Quintero_Into.pdf?sequence=3}

\end{thebibliography}

\end{document}
